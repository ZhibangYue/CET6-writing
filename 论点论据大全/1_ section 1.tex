\graphicspath{{Images/}}

\section{总览}
\subsection{从题型说起}
    六级写作主要是议论文(当然也存在极少数的应用文),具体可分为以下几类:
\begin{itemize}
    \item 意义措施类
    \item 观点选择类
    \item 现象解释类
    \item 谚语警句类
    \item 图画图表类
\end{itemize}
    \par 
    可以搜索PYY老师去全面地学习,或者直接点击链接\href{https://zhuanlan.zhihu.com/p/357293671}{\textit{六级写作模板170+保姆教程}}。
    \par
    这里只讨论论点和论据。
\subsection{论据的意义}
通常来说,一般都是类似于“五段三论”的写法,从现象说到意义或原因,最后提出措施。中间部分一般是全文的主题,是需要集中论证的地方。常用的论证方法有:
\begin{itemize}
    \item 引用论证
    \item 举例论证
    \item 对比论证
    \item 道理论证
    \item ...
\end{itemize}
\par
引用论证和举例论证是最好用,也最拉风的————因为它们说理逻辑性弱,不容易出错,而且能够体现作者的文化素养,有逼格。所以一般来说,建议主要采用这两种论证方式。
\par
既然如此,论据就显得很重要了。除了日常生活中的见闻和已有的知识储备,下面的这些或许也能有所帮助。
\subsection{额外的一点}
其实,如果研究过CET6写作的例文,会发现对比论证也占有一席之地。虽然这并非本书讨论的重点,但是值得提及。
\par
最简单的用法就是正反对比,即有xxx会怎么样,没有xxx会怎么样,具体可以采用举例论证、引用论证、演绎论证填充这个“框架”的血肉。
\subsection{推而广之}
既然如此,可以发现,上述道理对于简单的议论文写作是通用的。所以它不仅可以应用于CET6,还有考研英一英二等等……