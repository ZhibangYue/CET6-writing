\section{历史与文化}
\subsection{文化保护}
故宫博物院与腾讯合作推出了微信小游戏《故宫:口袋宫匠》,实现了电子游戏与传统文化的双向赋能,成为文物的数字化保存和保护的典范。但是这个游戏没有英文版,名字凭感觉翻译的...
\par
在这里可以发现,当下凡是跟文化相关的热点/新闻等(包括下述),都与科技和创新密不可分(它也适用于创新话题)。
\begin{itemize}
    \item The Palace Museum and Tencent have collaborated to launch the wechat-minigame "Gugong: Pocket Palace Craftsman", aimed at disseminating and promoting traditional culture. This initiative not only serves to popularize cultural heritage but also seeks to digitize the preservation and protection of cultural relics. 
\end{itemize}

\subsection{文化传播}
《原神》,中国文化出海的标兵。
\begin{itemize}
    \item  Genshin elevates Chinese cultural elements onto the global stage, successfully representing the richness of Chinese culture and reshaping the perception of 'made-in-China.' It has garnered widespread acclaim from players around the world.
\end{itemize}
\par
其实这也体现了中国传统文化的影响力、当代中国文化出口的成功等多个方面。
\subsection{中华优秀传统文化}
通常来说,这里细点比较多,因为中华文化博大精深(泱泱中华,万古长河。晨禹迹而暮朝歌,泽丰镐而卫河洛……),列举几个有代表性的:
\begin{itemize}
    \item Beijing Opera
    \item paper-cutting
    \item calligraphy
    \item landscape painting
    \item Tang poems
\end{itemize}
\par
下面这一段取材自中国日报,习近平总书记呼吁广大哲学社会科学工作者共同努力,在新的时代条件下推动中华优秀传统文化创造性转化、创新性发展。
\par
President Xi Jinping has called on people who work in the field of philosophy and social sciences to promote the evolution and growth of fine traditional Chinese culture in new and creative ways in the new era.
\subsection{文化创新}
文化创新是一个相当宽泛的话题,案例包括:
\begin{itemize}
    \item Gugong
    \item Genshin
    \item TV program National Treasure
\end{itemize}
\par
除了前述的两款游戏,还包括一些电视节目,例如《国家宝藏》、《典籍里的中国》等等(不再列举了),都是文化创新的典范,基本够用了(是的,只要是当代的事情,都离不开科技创新)。
\par
对策则有:
\begin{itemize}
    \item discard their dross and select their essence 
    \item critical inheritance
    \item combination of traditional culture, the new media and Internet technology
\end{itemize}
\par
取其精华,去其糟粕,批判继承,再加上结合新媒体和互联网技术。这些不仅是对策,也是文化创新成功的原因。另外,这里词性没有统一。