\graphicspath{{Images/}}

\section{以人为本}
以人为本是非常重要的思想。
\subsection{人才是第一资源}
滴滴底层软件故障,被认为是裁员潮带来的恶果。
\par
Meta通过招揽人才,即使多到花钱让他们做假工作,也不让其他公司得到他们。
\begin{itemize}
    \item In November 2023, Didi Global ride-hailing app was disrupted by an underlying system software failure, which is believed to be the impact of layoffs at the beginning of the year.————取材自网易、CNA
    \item According to Britney Levy, an ex-employee of Meta, this company paid employees to do "fake work" so that other companies couldn't have them.————取材自TikTok\cite{metalayouts}、Business Insider
\end{itemize}
\par
Meta的故事能在小视频里看到,可以去搜一搜,虽然热度已经过去了。
\subsection{教育是国之大计}
孔子“有教无类”的教育理念,使他“万世师表”实至名归。
\begin{itemize}
    \item Confucius was the first educator in Chinese history to put forward the idea of teaching without discrimination, giving everyone an equal opportunity to learn, which makes him the most renowned educator.
\end{itemize}
\subsection{生活与工作}
过了不到15年,同一批人觉得工作其实没有那么重要。
\begin{itemize}
    \item Research from King's College London based on surveys from 24 countries found that just 14\% of UK millennials believe work should always come first, compared with 41\% in 2009. Prof Bobby Duffy, the principal investigator in the study, suggested this might mark a cultural shift.————取材自The Guardian
\end{itemize}
\subsection{论人的自我修养}
这里就很多了,只要用名人就可以,而且基本具有普适性,也就是说不论是勤奋努力、矢志创新、精忠报国、勇敢无畏、自尊自信、坚持不懈等等等,不同的人物、不同的属性,都可以乱堆。
\par
因为太通俗了,所以不展开了。
\par
我还记得上次写作我写的就是Elon Musk,不过主题似乎是“辩证思考+理性选择”,emmm我个人倒是感觉也很合理。
\begin{itemize}
    \item Yuan Longpin and his hybrid rice
    \item Elon Musk and SpaceX / Telsa
    \item ...
\end{itemize}
\par 
在这里放一句可以用作“勇于尝试”或者“远大理想”主题的话,我很喜欢这一句:
\par
Shoot for the moon. Even if you miss, you'll land among the stars.
\subsection{人与自然}
人与自然是一个很宏大的视角,即使把人剥离出去,单纯讨论“自然”都必然会有长篇大论。
\par
IPCC(Intergovernmental Panel on Climate Change。即
联合国政府间气候变化专门委员会)的报告最具有权威性。下面这段节选自其2022年的报告\textit{Impacts, Adaptation and Vulnerability}的第一章,无修改。
\par
Numerous additional significant climate-related changes have unfolded worldwide. Consistent with projections, multiple concurrent changes in the physical climate system have grown more salient, including increasing global temperatures, loss of ice volume, rising sea levels and changes in global precipitation patterns. The changes in the physical climate system, most notably more intensive extreme events, have adversely affected natural and human systems around the world. This has contributed to a loss and degradation of ecosystems, including tropical coral reefs; reduced water and food security; increased damage to infrastructure; additional mortality and morbidity; human migration and displacement; damaged livelihoods; increased mental health issues; and increased inequality.
\par
其中所列写的具体环境问题有以下几条:
\begin{itemize}
    \item increasing global temperatures
    \item loss of ice volume
    \item rising sea levels
    \item reduced water and food security
    \item increased damage to infrastructure
    \item ...
\end{itemize}
\par
即使上面那段太长了难以使用,核心要点还是很明了的。
\par
至于对策,可以去详细阅读IPCC的报告,此外这一句很不错:
\par
Lucid waters and lush mountains are invaluable assets.
\par
绿水青山就是金山银山。
\subsection{健康最重要}
这部分是承接上一部分的,环境恶劣危害人的健康等等……
\par
举一个例子,新冠后遗症\textbf{post-COVID-19 condition}
\par
The World Health Organization has developed a definition for post-COVID-19 condition (commonly known as long COVID) as continuation or development of new symptoms 3 months after the initial SARS-CoV-2 infection, with these symptoms lasting for at least 2 months with no other explanation.
\par
While common symptoms of long COVID can include fatigue, shortness of breath and cognitive dysfunction over 200 different symptoms have been reported that can have an impact on everyday functioning.
\par
使用这个例子很有意义,因为它的应对措施很具有普适性,可以用在“提出措施”的部分(这也可以做论据啊)。
\begin{itemize}
    \item Taking up offers of COVID-19 vaccines/boosters
    \item Wearing well-fitted masks
    \item Cleaning hands regularly
    \item Catching coughs and sneezes
    \item Ensuring indoor spaces are well ventilated.
\end{itemize}
\par
以上都来自WHO官网,可以自己去搜索资料深入研究。早睡早起多喝热水等万用措施不再赘述。
\subsection{希望之花永不凋零}
除了《肖申克的救赎》中的点睛之笔,下面的句子也极为合适(来自二十年前已故的哲学家、学者伯纳德·威廉姆斯)。
\begin{itemize}
    \item There was never a night or a problem that could defeat sunrise or hope. ———— Bernard Williams
\end{itemize}
\subsection{人与人}
每个人都不是一座孤岛,因此人与人之间的距离与联系也是很有讨论意义的。具体包括朋友、家人、师生等等。
\par
下面这句话评价很到位:
\par
We're born into relationship, and it's in relationship that we find healing and growth and potential.———— Bruce Crapuchettes
\par
可以去这位博士的网站看看:\href{https://pasadenainstitute.com/}{https://pasadenainstitute.com/}
\par 
关于朋友:
\par
Is it not delightful to have friends coming from distant quarters?———— The Analects
\par 
有朋自远方来,不亦乐乎?————《论语·学而》
\par
其他的也都是老生常谈了,不再列举了。